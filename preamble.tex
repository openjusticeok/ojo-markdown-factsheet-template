% explicitly call these packages to avoid this issue:
% https://stackoverflow.com/questions/46080853/why-does-rendering-a-pdf-from-rmarkdown-require-closing-rstudio-between-renders?
% utm_medium=organic&utm_source=google_rich_qa&utm_campaign=google_rich_qa
\usepackage{booktabs}
\usepackage{longtable}
\usepackage{array}
\usepackage{multirow}
\usepackage{color}          % enables font colors
\usepackage{wrapfig}
\usepackage{float}
\usepackage{colortbl}
\usepackage{pdflscape}
\usepackage{tabu}
\usepackage{graphicx}
\usepackage{threeparttable}
\usepackage[normalem]{ulem}
\usepackage[export]{adjustbox}  % enables two column images at the top
\usepackage{makecell}           % for library(kableExtra)
%\usepackage{tabularx}           % for library(kableExtra)
\usepackage{fancyhdr}           %includes fancy header package
\usepackage[most]{tcolorbox}

% set font encoding for PDFLaTeX or XeLaTeX
\usepackage[utf8]{inputenc}   % sets document encoding to utf8
\usepackage[default]{lato}    % sets default font to Lato
\usepackage[T1]{fontenc}
%\usepackage{fontspec}         %Font package
%\setmainfont{Pluto Sans Cond Bold}  
\usepackage{tikz}
\usetikzlibrary{calc}
\usepackage{memhfixc}
\usepackage{amssymb}          % for itemized list styles

% document setup
\definecolor{urbnblue}{HTML}{1696D2}    % defines the Urban Institute blue color
\usepackage{enumitem}                   % bullet alignment
\setlist[2]{nosep}                      % sets the itemsep and parsep for all level two lists to 0
\setenumerate{nosep}                    % sets no itemsep for enumerate lists only
\usepackage[hang,flushmargin]{footmisc} % don't indent footnotes
\usepackage[document]{ragged2e}         % ragged right
\usepackage[none]{hyphenat}             % disable work breaks
\renewcommand\thefootnote{\textcolor{urbnblue}{\arabic{footnote}}} % change the color of the footnote indicator
\hypersetup{
  colorlinks,
  linkcolor=urbnblue,
  urlcolor=urbnblue,
}
\usepackage{parskip}
\setlength{\parskip}{0.1in}               % set paragraph spacing
\pagestyle{fancy}


%trying color box
\newtcolorbox{mybox}{colback=grey!5!white}

% % Open Justice Oklahoma logo
% \newcommand{\ojologo}[0]{
%   \begin{figure}
%     \begin{flushright}
%       \includegraphics[width=1.5in,keepaspectratio]{design/cropped-OJO_Logo_Black.jpg}
%     \end{flushright}
%   \end{figure}
% }

% Open Justice Oklahoma logo
\newcommand{\ojologo}[0]{
  \begin{wrapfigure}{r}{.25\textwidth}
      \includegraphics[width=2in,keepaspectratio]{design/cropped-OJO_Logo_Black.jpg}
  \end{wrapfigure}
}

% contact information
\newcommand{\footer}[0]{
  \thispagestyle{empty}
  \begin{tikzpicture}[remember picture, overlay]
    \draw 
      node[anchor=south west, minimum height=0.5in, minimum width=8.5in, text width=8in] at ($(current page.south west) + (0in,0in)$) 
      {\thepage \space \rule{485pt}{3pt}\textcolor{black}{~OpenJusticeOK.org}};
  \end{tikzpicture}
}

% titles (14pt font)
\newcommand{\ojotitle}[1]{
      \textbf{\LARGE{#1}}
}
  
% subtitles (12pt font)
\newcommand{\ojosubtitle}[1]{
  \begin{flushleft}
    \Large{{#1}}    
  \end{flushleft}
}

% authors (11pt font)
\newcommand{\ojoauthors}[1]{
  \begin{center}
    \textit{\large{#1}}
    \vspace{0.25in} 
  \end{center}
}

%header for 1st page
\newtcolorbox{ojomainheader}[1][]{%
  enhanced,
  frame hidden,
  colback=white,
  left=0pt,
  nobeforeafter,
  top=-4pt,
  bottom=-7pt,
  borderline west={5pt}{0mm}{yellow},
  #1
}

%heading command
\newtcbox{heading}[1][]{%
  enhanced,
  frame hidden,
  colback=white,
  fontupper=\bfseries \Large,
  left=0pt,
  nobeforeafter,
  top=-4pt,
  bottom=-6.5pt,
  borderline west={5pt}{0mm}{yellow},
  #1
}

% heading 2
\newcommand{\urbnheadingtwo}[1]{
  \textbf{\textcolor{urbnblue}{#1}}
}

% figure label
\newcommand{\urbnfigurenumber}[1]{
  \textcolor{urbnblue}{\tiny{\\} \normalsize{FIGURE #1 \\}}
}

% figure title
\newcommand{\urbnfiguretitle}[1]{
  \textbf{\normalsize{#1} \\}
}

% figure source
\newcommand{\urbnsource}[1]{
  \footnotesize{\textbf{Source: }#1} \normalsize\break
}

% figure note
\newcommand{\urbnnote}[1]{
  \footnotesize{\textbf{Note: }#1} \normalsize\break
}

% table label
\newcommand{\urbntablenumber}[1]{
  \textcolor{urbnblue}{\tiny{\\} \normalsize{TABLE #1 \\}}
}

% table title
\newcommand{\urbntabletitle}[1]{
  \textbf{\normalsize{#1} \\}
}

% table sub title
\newcommand{\urbntablesubtitle}[1]{
  \textit{\normalsize{#1} \\}
}

% bullet points
\renewcommand{\labelitemi}{\color{urbnblue}\tiny$\blacksquare$}   % blue square
\renewcommand{\labelitemii}{\color{urbnblue}\tiny$\blacksquare$}  % blue square
\renewcommand{\labelitemiii}{\color{urbnblue}\tiny$\blacksquare$} % blue square
\renewcommand{\labelitemiv}{\color{urbnblue}\tiny$\blacksquare$}  % blue square

\newenvironment{urbnbullets}
{\begin{itemize}[leftmargin=*,labelindent=0.25in,labelsep=0.1875in]
\setlength{\itemsep}{0pt}
\setlength{\parskip}{0pt}
\setlength{\parsep}{0pt}}
{\end{itemize}} 

% numbered list
\newenvironment{urbnenumerate}
{\begin{enumerate}[leftmargin=*,labelindent=0.25in,labelsep=0.15625in]
\setlength{\itemsep}{0pt}
\setlength{\parskip}{0pt}
\setlength{\parsep}{0pt}}
{\end{enumerate}} 

% boiler plate
\newcommand{\urbnboilerplate}[3]{
  \thispagestyle{empty}
  
  \begin{tikzpicture}[remember picture, overlay]
    % horizontal line
    \draw (current page.south west) + (1in,1.25in) -- ($(current page.south east) + (-1in, 1.25in)$);  
    % boilerplat language
    \draw 
      node[align=left, anchor=south west, minimum height=0.5in, minimum width=8.5in, text width=6.5in] at ($(current page.south west) + (0in,0.38in)$) 
      {
      \small{This fact sheet was funded by #1. The views expressed are those of the author/authors and should not be attributed to the Urban Institute, its trustees, or its funders. Further information on the Urban Institute’s funding principles is available at \href{https://www.urban.org/aboutus/our-funding/funding-principles}{urban.org/fundingprinciples.}
      
      
      Copyright © #2 #3. Urban Institute. Permission is granted for reproduction of this file, with attribution to the Urban Institute.}
      };
      \end{tikzpicture}
}

% eliminates top border line for the table
\def\toprule{}

% sets midline at 1pt
\setlength{\lightrulewidth}{1pt}